\section{Simulating the solar system}
The exercise is done in the script solar$\_$ss.py. The necessary explanations of the methods used are in the comments of the code. 
For question (a), we do the following: \lstinputlisting[firstline=1,lastline=47]{solar_ss.py} 
The resulting plot is in Fig. \ref{fig:fig1a}.

\begin{figure}[h!]
    \centering
    \includegraphics[width=0.9\linewidth]{./plots/fig1a.png}
    \caption{Plot showing the initial conditions generated for the Solar System, in UA units. Left panel: $(x,y)$ positions at the current time. Right panel: $(x,z)$ positions at the current time.}
    \label{fig:fig1a}
  \end{figure}

For question (b), 
we do the following: \lstinputlisting[firstline=49,lastline=150]{solar_ss.py}
The resulting plot is in Fig. \ref{fig:fig1b}, while a zoom on the inner planets is found at Fig. \ref{fig:fig1bzoom}.
% aggiungi: Explain whether leapfrog is a suitable choice of an algorithm, and why.

\begin{figure}[h!]
    \centering
    \includegraphics[width=0.9\linewidth]{./plots/fig1b.png}
    \caption{Plot showing the orbits of the planets in the Solar System, using Leapfrog method.
    The positions of the objects are computed over a time of 200 years and are expressed in UA units. Right panel: $(x,y)$ positions over time. Left panel: $(z)$ coordinate over time.} 
    \label{fig:fig1b}
  \end{figure}

  \begin{figure}[h!]
    \centering
    \includegraphics[width=0.9\linewidth]{./plots/fig1b_zoom.png}
    \caption{Zoom showing the orbits of the inner planets in the Solar System, using Leapfrog method.
    The positions of the objects are computed over a time of 200 years and are expressed in UA units. Right panel: $(x,y)$ positions over time. Left panel: $(z)$ coordinate over time.} 
    \label{fig:fig1bzoom}
  \end{figure}

For question (c), 
we do the following: \lstinputlisting[firstline=152,lastline=233]{solar_ss.py}
%aggiungi: compare the results you get to those obtained with the leapfrog method
The resulting plot is in Fig. \ref{fig:fig1c}. The difference in the $x$ coordinates computed by the two methods is found at Fig. \ref{fig:fig1cdiff}.

\begin{figure}[h!]
    \centering
    \includegraphics[width=0.9\linewidth]{./plots/fig1c.png}
    \caption{Plot showing the orbits of the planets in the Solar System, using Runge-Kutta method.
    The positions of the objects are computed over a time of 200 years and are expressed in UA units. Right panel: $(x,y)$ positions over time. Left panel: $(z)$ coordinate over time.} 
    \label{fig:fig1b}
  \end{figure}

  \begin{figure}[h!]
    \centering
    \includegraphics[width=0.9\linewidth]{./plots/fig1c_diff.png}
    \caption{Difference $(z)$ coordinate over time (100 years in this case, for clearer visualization) between the Leapfrog and Runge-Kutta methods.} 
    \label{fig:fig1bzoom}
  \end{figure}

\section{Calculating forces with the FFT}
The exercise is done in the script fft.py. The necessary explanations of the methods used are in the comments of the code.
For question (a), we do the following: \lstinputlisting[firstline=4,lastline=57]{fft.py}
The resulting plot is in Fig. \ref{fig:fig2a}.

\begin{figure}[h!]
  \centering
  \includegraphics[width=0.9\linewidth]{./plots/fig2a.png}
  \caption{Colormaps of the 2D slices of the grid at $z$ = 4.5, 9.5, 11.5 and 14.5, showing the $\delta$ assigned to each grid point.} 
  \label{fig:fig2a}
\end{figure}

For question (b), the code is below: \lstinputlisting[firstline=59,lastline=213]{fft.py}
The plot for the potential $\phi$ is in Fig. \ref{fig:fig2b}, while the log of the absolute value of the Fourier-transformed potential is in Fig. \ref{fig:fig2blog}.

\begin{figure}[h!]
  \centering
  \includegraphics[width=0.9\linewidth]{./plots/fig2b.png}
  \caption{Colormap of the potential $\phi$ at $z$ = 4.5, 9.5, 11.5 and 14.5.} 
  \label{fig:fig2b}
\end{figure}

\begin{figure}[h!]
  \centering
  \includegraphics[width=0.9\linewidth]{./plots/fig2b_pot.png}
  \caption{Colormap of the log of the absolute value of the Fourier-transformed potential at $z$ = 4.5, 9.5, 11.5 and 14.5.} 
  \label{fig:fig2blog}
\end{figure}

\section{Spiral and elliptical galaxies}
The code for part (a) of this exercise is shown below: \lstinputlisting[firstline=5,lastline=54]{learning.py}
The resulting plot is in Fig. \ref{fig:fig3a}.
The output of the first 10 rescaled features is in: \lstinputlisting{3a.txt}

\begin{figure}[h!]
  \centering
  \includegraphics[width=0.9\linewidth]{./plots/fig2b_pot.png}
  \caption{Distributions of the rescaled features, having mean 0 and standard deviation 1.} 
  \label{fig:fig3a}
\end{figure}

For part (b), the code follows: \lstinputlisting[firstline=56,lastline=221]{learning.py}
The resulting plot is in Fig. \ref{fig:fig3b}.
%comment

\begin{figure}[h!]
  \centering
  \includegraphics[width=0.9\linewidth]{./plots/fig3b.png}
  \caption{Evolution of the cost function convergence with the number of iterations, as result of the minimisation routine applied to the two sets of features.} 
  \label{fig:fig3b}
\end{figure}

For part (c), we do the following: \lstinputlisting[firstline=223,lastline=280]{learning.py}
The plot we obtained is in Fig. \ref{fig:fig3c}.
The number of true/false positives/negatives, as well as the F1 score can be seen in: \lstinputlisting{3c.txt}

\begin{figure}[h!]
  \centering
  \includegraphics[width=0.9\linewidth]{./plots/fig3c.png}
  \caption{Scatter plots showing each pair of features against each other, with the decision boundary as result of the logistic regression.} 
  \label{fig:fig3c}
\end{figure}
%comment


