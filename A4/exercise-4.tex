\section{Simulating the solar system}
The exercise is done in the script solar_ss.py. The necessary explanations of the methods used are in the comments of the code. 
For question (a), we do the following: \lstinputlisting[firstline=1,lastline=47]{solar_ss.py} 
The resulting plot is in Fig. \ref{fig:fig1a}.

\begin{figure}[h!]
    \centering
    \includegraphics[width=0.9\linewidth]{./plots/fig1a.png}
    \caption{Plot showing the initial conditions generated for the Solar System, in UA units. Left panel: $(x,y)$ positions at the current time. Right panel: $(x,z)$ positions at the current time.}
    \label{fig:fig1a}
  \end{figure}

For question (b), 
we do the following: \lstinputlisting[firstline=49,lastline=150]{solar_ss.py}
The resulting plot is in Fig. \ref{fig:fig1b}, while a zoom on the inner planets is found at Fig. \ref{fig:fig1bzoom}.
% aggiungi: Explain whether leapfrog is a suitable choice of an algorithm, and why.

\begin{figure}[h!]
    \centering
    \includegraphics[width=0.9\linewidth]{./plots/fig1b.png}
    \caption{Plot showing the orbits of the planets in the Solar System, using Leapfrog method.
    The positions of the objects are computed over a time of 200 years and are expressed in UA units. Right panel: $(x,y)$ positions over time. Left panel: $(z)$ coordinate over time.} 
    \label{fig:fig1b}
  \end{figure}

  \begin{figure}[h!]
    \centering
    \includegraphics[width=0.9\linewidth]{./plots/fig1b_zoom.png}
    \caption{Zoom showing the orbits of the inner planets in the Solar System, using Leapfrog method.
    The positions of the objects are computed over a time of 200 years and are expressed in UA units. Right panel: $(x,y)$ positions over time. Left panel: $(z)$ coordinate over time.} 
    \label{fig:fig1bzoom}
  \end{figure}

For question (c), 
we do the following: \lstinputlisting[firstline=152,lastline=233]{solar_ss.py}
%aggiungi: compare the results you get to those obtained with the leapfrog method
The resulting plot is in Fig. \ref{fig:fig1c}. The difference in the $x$ coordinates computed by the two methods is found at Fig. \ref{fig:fig1cdiff}.

\begin{figure}[h!]
    \centering
    \includegraphics[width=0.9\linewidth]{./plots/fig1c.png}
    \caption{Plot showing the orbits of the planets in the Solar System, using Runge-Kutta method.
    The positions of the objects are computed over a time of 200 years and are expressed in UA units. Right panel: $(x,y)$ positions over time. Left panel: $(z)$ coordinate over time.} 
    \label{fig:fig1b}
  \end{figure}

  \begin{figure}[h!]
    \centering
    \includegraphics[width=0.9\linewidth]{./plots/fig1c_diff.png}
    \caption{Difference $(z)$ coordinate over time (100 years in this case, for clearer visualization) between the Leapfrog and Runge-Kutta methods.} 
    \label{fig:fig1bzoom}
  \end{figure}

\section{Calculating forces with the FFT}